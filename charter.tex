\documentclass{article}
\usepackage[utf8]{inputenc}
\usepackage[russian]{babel}
\usepackage[a4paper, total={7in, 9in}]{geometry}
\usepackage{enumitem}
\usepackage{titlesec}
\usepackage{amsmath}
\usepackage[bookmarks=true]{hyperref}
\usepackage{bookmark}
\setlength{\parindent}{0em}
\setlength{\parskip}{1em}
\newcounter{artctr}

\newenvironment{art}[1]{%      define a custom environment
   \bigskip\noindent%         create a vertical offset to previous material
   \refstepcounter{artctr}% increment the environment's counter
   \pdfbookmark{\theartctr. #1}{\theartctr}
   \begin{center}
       \large{\textbf{\theartctr. #1}}
   \end{center}
   \nopagebreak
   \begin{enumerate}
   }{
   \end{enumerate}
   }  %          create a vertical offset to following material

\title{\vspace{-5ex}{Устав Клуба Дебатов МФТИ}\\\vspace{0.5cm}\normalsize\textit{принятый Организационным комитетом Клуба Дебатов МФТИ ?? марта 2020 года}\vspace{-2cm}}
\date{}

\begin{document}

\maketitle
\begin{art}{Общие положения}
\item Клуб Дебатов МФТИ (дальше -- Клуб) является добровольным объединением студентов, аспирантов, выпускников и сотрудников МФТИ (НИУ), действующих и недействующих Почётных членов Клуба, а также иных лиц, принимающих участие в деятельности и организации Клуба, составляющих сообщество Клуба Дебатов МФТИ (дальше -- сообщество Клуба).

\item Цель клуба -- развитие культуры дебатов в МФТИ.

\item Управление деятельностью Клубом осуществляется (в порядке убывания приоритета решений) Организационным Комитетом Клуба и Председателем Клуба в рамках, определённых настоящим Уставом.

\item Названия Клуба:
\begin{itemize}[topsep=0pt]
    \item по-русски: Клуб Дебатов МФТИ
    \item по-английски: MIPT Debate Club
\end{itemize}

\item Клуб в своей деятельности руководствуется настоящим Уставом, Уставом МФТИ и законодательством Российской Федерации.

\item Клуб сотрудничает с администрацией МФТИ, другими дебатными клубами, прочими организациями и объединениями на условиях взаимного уважения и независимости.

\item Деятельность Клуба приостанавливается на время зачётной недели, экзаменационной сессии, каникул и праздничных дней.

\item Клуб не допускает ущемления прав и свобод человека, дискриминации по половому, расовому или любому другому признаку. Язык вражды недопустим в рамках мероприятий Клуба.

\item Деятельность Клуба основана на принципах открытости и демократии. Клуб стремится к созыву Собрания членов Клуба, постоянно действующего представительного органа управления Клубом, путём поправок к Уставу.

\end{art}

\begin{art}{Организационный комитет}
\item Организационный комитет (дальше -- Оргкомитет) занимается организацией деятельности Клуба.
\item Оргкомитет состоит из:
\begin{enumerate}
    \item Председателя Клуба
    \item Действующих Почётных членов Клуба
    \item Ответственных:
    \begin{enumerate}
        \item за финансы
        \item за документы
        \item за социальные медиа
        \item за дизайн
        \item за взаимодействие с иностранными студентами
    \end{enumerate}
    \item Советников Клуба
\end{enumerate}
\item Решения Оргкомитета принимаются простым большинством проголосовавших, если не оговорено противного. Порядок учёта голосов в рамках онлайн-голосования при необходимости уточняется решением Оргкомитета.
\item Оргкомитет регулярно проводит открытые встречи для публичного обсуждения вопросов организации Клуба.
\item Все решения Оргкомитета, представляющие общественный интерес или регулирующие деятельность сообщества Клуба, подлежат своевременной публикации на онлайн-ресурсах Клуба.
\item Члены Оргкомитета имеют один голос во время голосований.
\item Члены Оргкомитета могут занимать несколько должностей, определённых Уставом.
\end{art}

\begin{art}{Председатель Клуба}
\item Председатель Клуба выбирается Оргкомитетом из числа студентов, аспирантов или сотрудников МФТИ.
\item Председатель Клуба 
\begin{itemize}
    \item делегирует свои полномочия другим членам Оргкомитета и активистам Клуба по своему усмотрению
    \item представляет Клуб перед администрацией МФТИ, другими дебатными клубами, прочими организациями и объединениями
    \item управляет расписанием и программой Клуба
    \item координирует работу Оргкомитета
    \item ведёт список состава Оргкомитета, обновляя его в соответствии с решениями Оргкомитета
    \item проводит публичный отбор кандидатур ответственных и советников Клуба и вносит их на рассмотрение Оргкомитетом
    \item поддерживает дружественную и продуктивную обстановку в сообществе Клуба и Оргкомитете
    \item переводит действующих Почётных членов Клуба в недействующих в случае их неучастия в деятельности Клуба
    \item хранит и распоряжается имуществом Клуба
\end{itemize}
\item Срок полномочий Председателя Клуба составляет два года. Одно и то же лицо не может занимать должность Председателя Клуба более двух сроков.
\item Треть членов Оргкомитета или один Почётный член могут выразить недоверие Председателю Клуба, после чего проходят внеочередные выборы.
\item Если Председатель Клуба был отчислен или уволен из МФТИ, ушёл в академический отпуск, либо по любым другим причинам неспособен выполнять свои обязательства, то он уходит в отставку.
\item Председатель Клуба может добровольно подать в отставку, после чего проходят внеочередные выборы.
\item Если Председатель Клуба не отстранён от своего поста в результате выражения недоверия и последующих внеочередных выборов, Председатель Клуба после окончания своего срока становится действующим Почётным членом Клуба.
\end{art}

\begin{art}{Почётные члены Клуба}
\item Почётные члены Клуба обеспечивают долгосрочную преемственность традиций управления Клуба.
\item Почётные члены делятся на действующих и недействующих.
\item Действующие Почётные члены способствуют разрешению конфликтов на предмет соответствия действующему Уставу. Если стороны конфликта не могут прийти к решению в течении недели, действующие Почётные члены принимают решение простым большинством действующих Почётных членов. Действующие Почётные члены не могут быть лишены своего статуса действующих Почётных членов во время разрешения конфликтов.
\item Почётные члены могут перейти в разряд действующих или недействующих путём уведомления Председателя Клуба.
\item Почётное Членство является добровольным. Лица, обладающие данным статусом, могут от него отказаться путём уведомления Председателя Клуба.
\item Действующие Почётные члены могут перейти в разряд недействующих путём уведомления Председателя Клуба.
\item Почётный член Клуба, не соответствующий своему высокому званию или злоупотребляющий им, может быть лишён своего статуса решением двух третей всего состава Оргкомитета без Почётных членов.
\end{art}

\begin{art}{Ответственные и советники Клуба}
\item Ответственные и советники Клуба назначаются из числа сообщества Клуба. Действующие почетные члены не могут быть ответственными или советниками Клуба. Члены Оргкомитета не могут голосовать за назначение себя на какую-либо должность или отстранение с неё. Предложения о назначении и отстранении вносятся на голосование Председателем Клуба.
\item Ответственные и советники Клуба могут сложить с себя полномочия по собственному желанию путём уведомления Председателя Клуба.
\item Полномочия и обязанности ответственных и советников Клуба определяются Председателем Клуба с учётом мнения членов Оргкомитета и согласно предшествующим решениям Оргкомитета и положениям Устава.
\end{art}

\begin{art}{Поправки и хранение Устава}
\item Предложение о поправках могут вносить на рассмотрение Оргкомитетом члены Оргкомитета.
\item Предложения о поправках, не затрагивающие главы 1 и 6, считаются принятыми, если они одобрены двумя третями всего состава Оргкомитета и подписаны Председателем Клуба.
\item Предложения о поправках, затрагивающие главы 1 и 6, могут быть рассмотрены в рамках основательного пересмотра всего устава Собранием членов Клуба, созыв которого регулируется и объявляется решением Оргкомитета.
\item Устав и поправки к нему хранятся публично в электронном виде на онлайн-ресурсах Клуба и в бумажном виде в Службе поддержки студенческих инициатив Управления внеучебной деятельностью МФТИ.
\end{art}

\newpage
\pdfbookmark{Приложение №1. Переходные положения}{\theartctr}
\begin{flushright}
    Приложение №1 к Уставу
\end{flushright}
\begin{center}
   \large{\textbf{Переходные положения}}
\end{center}
\begin{enumerate}
    \item После принятия настоящего Устава, все бывшие Председатели до 2019 года становятся Почётными членами Клуба. Джамиль Закиров становится действующим Почётным членом, прочие бывшие Председатели Клуба становятся недействующми Почётными членами:
    \begin{enumerate}
        \item Андрей Мухин
        \item Дмитрий Устинов
        \item Илья Черкасский
        \item Сергей Базылик
        \item Алексей Чекмарев
        \item Руслан Талипов
    \end{enumerate}
    \item Действующим Председателем Клуба является Илья Захаров. Его срок Председательства оканчивается 1 августа 2020 года.
    \item Действующими членами Оргкомитета после принятия настоящего Устава являются:
    \begin{itemize}
        \item Илья Захаров, Председатель Клуба
        \item Джамиль Закиров, действующий Почётный член Клуба
        \item Илья Удовин, советник Клуба
        \item Альфред Ильясов, советник Клуба
        \item Сергей Тихомиров, советник Клуба
        \item Сатиш-Чандра Читрапу, советник Клуба
        \item Надежда Дмитриенко, ответственная за социальные медиа
    \end{itemize}
\end{enumerate}
\end{document}